% Simpozijum Matematika i primene
% Requires Latex2e!
\documentclass[cyr]{simposium}


\volume{,Vol. X}  %%%%%% UPISUJE UREDNIK
\issue{(1)}        %%%%%% UPISUJE UREDNIK
\pubyear{2019}     %%%%%% UPISUJE UREDNIK
\firstpage{1}      %%%%%% UPISUJE UREDNIK
\lastpage{12}      %%%%%% UPISUJE UREDNIK

%%%%%% NA OVOM MESTU UKLJUCUJETE PAKETE
%%%%%% Na primer: \usepackage{gclc}

%%%%%% NA OVOM MESTU DEFINISETE LATEX KOMANDE
%%%%%% Na primer: \newcommand{\const}{\mathop{\mathrm{const}}}




\begin{document}
\begin{frontmatter}

\title{{\Lat GeoDemonstrator}}

\author{{\fnms{Lazar} \snm{Vasovi\cc}}}
\address{Matemati\ch ki fakultet, Univerzitet u Beogradu, Student{}ski trg 16, Beograd\\
\email{mi16099@alas.matf.bg.ac.rs}}

\runningauthor{ L. Vasovi\cc}
\runningtitle{{\Lat GeoDemonstrator}}

\received{\smonth{{  Jul}} \syear{2013}}   %%%%%% UPISUJE UREDNIK
\revised{\smonth{{  Jul}} \syear{2013}}    %%%%%% UPISUJE UREDNIK
\accepted{\smonth{{  Jul}} \syear{2013}}   %%%%%% UPISUJE UREDNIK

\maketitle


\begin{abstract}
    {\Lat GeoDemonstrator} je aplikacija napisana sa ciljem da korisnika bli\zh e uputi u geometrijske transformacije u ravni. Zamisao je omogu\cc iti jednostavno interaktivno prikazivanje i lak\sh e razumevanje materije koja se, izmedju ostalog, obradjuje na nekoliko predmeta  na osnovnim akademskim studijama informatike (I smer) na Matemati\ch kom fakultetu Univerziteta u Beogradu. Medju njima su Geometrija (G), Ra\ch unarska grafika (RG), kao i Primena projektivne geometrije u ra\ch unarstvu (PPGR). U pitanju je slo\zh ena aplikacija nastala kroz nabrojane kurseve, kao i kurs Programske paradigme (PP), tako da se sastoji iz nekoliko delova koji \ch ine celinu. U prvom delu, korisnik se kroz interaktivni interpretator upoznaje sa matemati\ch kim osnovama geometrijskih transformacija. Podr\zh ane operacije su pravljenje ta\ch aka i preslikavanja, primena, kompozicija, inverzija i druge. U drugom delu, korisnik se upoznaje sa primenom afinih transformacija, prate\cc i pritom \sh ta se ta\ch no de\sh ava prilikom samog preslikavanja. Mi\sh em zadaje proizvoljnu figuru u dvodimenzionom okru\zh enju, koju zatim transformi\sh e na na\ch in zadat takodje pomo\cc u grafi\ch kog korisni\ch kog interfejsa. U tre\cc em i \ch etvrtom delu, korisnik se upoznaje sa primenom projektivnih preslikavanja, i to kroz problem otklanjanja perspektivne distorzije odnosno rektifikacije proizvoljne \ch etvorke ta\ch aka u op\sh tem polo\zh aju, kao i problem pravljenja panorame od preklapaju\cc ih slika. U\ch itava slike sa datote\ch nog sistema ra\ch unara, nakon \ch ega ih obradjuje na \zh eljeni na\ch in.
\end{abstract}
\begin{keyword}
   vizuelizacija preslikavanja; afino preslikavanje; projektivno preslikavanje; perspektivna distorzija; panorama.
\end{keyword}
\end{frontmatter}




\end{document}
