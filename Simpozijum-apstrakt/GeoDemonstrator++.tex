\documentclass[11pt]{article}
\usepackage{hyperref}
\usepackage[utf8]{inputenc}
\usepackage{geometry}
 \geometry{
 a4paper,
 margin=1.5in,
 top=10mm,
 }
\pagenumbering{gobble}

\title{GeoDemonstrator}
\date{}
\author{Lazar Vasovi\'c}

\begin{document}
\maketitle

GeoDemonstrator je aplikacija napisana sa ciljem da korisnika bli\v{z}e uputi u geometrijske transformacije u ravni. Ideja je omogu\'citi jednostavno interaktivno prikazivanje i lak\v{s}e razumevanje materije koja se obradjuje na \v{c}asovima Geometrije (G), Ra\v{c}unarske grafike (RG), kao i Primene projektivne geometrije u ra\v{c}unarstvu (PPGR), sve predmeta na osnovnim akademskim studijama informatike (I smer) na Matemati\v{c}kom fakultetu Univerziteta u Beogradu.

U pitanju je hibridna aplikacija nastala kroz nabrojane kurseve, kao i kurs Programske paradigme (PP), pa se sastoji iz nekoliko delova koji \v{c}ine celinu:
\begin{itemize}
\item U prvom delu, korisnik se kroz interaktivni interpretator upoznaje sa matema- ti\v{c}kim osnovana geometrijskih transformacija. Podr\v{z}ano je pravljenje ta\v{c}aka i transformacija, primena, kompozicija, inverzija i druge operacije.

\item U drugom delu, korisnik se upoznaje sa primenom afinih transformacija, prate\'ci pritom \v{s}ta se ta\v{c}no de\v{s}ava prilikom same transformacije. Mi\v{s}em zadaje proizvoljnu figuru u dvodimenzionom okru\v{z}enju, koju zatim transformi\v{s}e na na\v{c}in zadat takodje pomo\'cu grafi\v{c}kog korisni\v{c}kog interfejsa.

\item U tre\'cem i \v{c}etvrtom delu, korisnik se upoznaje sa primenom projektivnih transformacija, i to kroz problem otklanjanja perspektivne distorzije odnosno rektifikacije, kao i problem pravljenja panorame. U\v{c}itava slike sa datote\v{c}nog sistema ra\v{c}unara, nakon \v{c}ega ih obradjuje na \v{z}eljeni na\v{c}in.
\end{itemize}

\end{document}